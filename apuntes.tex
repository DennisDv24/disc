\documentclass[10pt,a4paper]{article}
\usepackage[spanish]{babel}
\usepackage[utf8]{inputenc}
\usepackage{amsmath}
\usepackage{amsfonts}
\usepackage{amssymb}

\usepackage{amsthm}
	\theoremstyle{definition}
	\newtheorem{definition}{Definición}[section]
	\newtheorem{theorem}{Teorema}[section]
	\newtheorem{obs}{Observación}[section]

\usepackage{xcolor}
\usepackage[margin=0.9in]{geometry}

\spanishdecimal{.}

\newcommand{\Mod}[1]{\ (\mathrm{m\acute od}\ #1)}

\title{Apuntes Discretas}
\author{Oskar Denis Siodmok}
\begin{document}
\maketitle
\part{Aritmética modular}
\section{Conceptos básicos}
\begin{definition}
\[a\equiv b \Mod{n} \iff a-b = \dot{n} \:\land\: a\bmod n = b\bmod n,\: n>1\]
\end{definition}

\begin{theorem}
\[\forall a \in\mathbb{Z}\:\exists\:b\in\{0,1,\dots,n-1\}\subseteq\mathbb{Z}\:/\:a\equiv b\Mod{n},\:n\in\mathbb{N}\setminus\{1\}\]
\end{theorem}

\begin{definition}
\[[a]_n = \{b\in\mathbb{Z}:a\equiv b\Mod{n}\}\]
\end{definition}

\begin{obs}
Se cumple, debido a la propiedad transitiva: 
\[a\equiv b\Mod{n}\iff[a]_n = [b]_n\]
\end{obs}

\begin{definition}
	\[\mathbb{Z}_n=\{[0]_n,[1]_n,[2]_n,\dots,[n-1]_n\}\]
\end{definition}

\begin{obs}
	\[[a]_n \longrightarrow a \equiv a+n \equiv a+2n \equiv \dots \Mod{n}\:\implies\]
\[\text{Se puede aplicar la propiedad transitiva para transformar congruencias en otras equivalentes más sencillas}\]
Por ejemplo:
\[x\equiv17\Mod{11}\iff x\equiv 17-11\Mod{11} \iff x\equiv 17\bmod 11\Mod{11}\iff x\equiv 6\Mod{11}\]
Siendo ésta última ecuación la solución para $x$.
\end{obs}

\begin{obs}
	\[a+c\equiv b+d\Mod{n}
	\Longleftarrow\left\{\begin{array}{c}
		a\equiv b\Mod{n}\\ 
		c\equiv d\Mod{n}
	\end{array}\right\}\Longrightarrow
	ac\equiv bd\Mod{n}\]
Demostración:
\[\left.\begin{array}{c}
	a-b = kn\\
	c-d = ln
\end{array}\right\}a+c-b-d = kn+ln = n(k+l)\implies a+c\equiv b+d\Mod{n}
\]
\[\left.\begin{array}{c}
	a-b = kn;\:ac-bc = kn\\
	c-d = ln;\:cb-db = ln
\end{array}\right\}ac+cb-bc-db = kn+ln= n(k+l) = ac-db \implies ac\equiv bd \Mod{n} \]
\end{obs}
\begin{definition}
	Debido a estas observaciones, se define la suma y producto de clases módulo n como:
\[
\begin{array}{c}
	[a]_n+[b]_n = [a+b]_n\\\relax
	[a]_n [b]_n = [ab]_n
\end{array}
\]
\end{definition}

\section{Congruencias lineales}
\begin{definition} Se define una congruencia lineal como:
	\[ax+b\equiv c \Mod{n};\: a,b,c\in\mathbb{Z}\]
\end{definition}

\begin{theorem}
	\[\text{mcd}(a,n) = 1 \iff \exists\: b\:/\:ab\equiv 1\Mod{n}\]
	Se dice que $\exists\:[a]_n^{-1}$ para el producto en $\mathbb{Z}_n$
\end{theorem}

\begin{definition}
	\[\begin{array}{rl}
		\phi:\mathbb{Z}^+\longrightarrow & \mathbb{Z}^+\\
		n\longmapsto & \phi(n) = \#\{m\in\mathbb{N}:\text{mcd}(m,n)=1,\: m\leq n\}\\&
		\phantom{\phi(n)}= \#\{m\in\mathbb{N}:m\text{ coprimo con }n,\:m\leq n\}
	\end{array}\]
\end{definition}

\begin{theorem}
	\[
		\begin{array}{l}
				\phi(p) = p-1\iff p\in\text{primos}\\
				\phi(p^\alpha) = p^\alpha-p^{\alpha-1}\:\forall\:\alpha\in\mathbb{N}\iff p\in\text{primos}\\
				\phi(mn)=\phi(m)\phi(n)\:\forall\:m,n\in\mathbb{Z}^+\iff\text{ mcd}(m,n)=1\\
					n = p_1^{\alpha_1}p_2^{\alpha_2}\dots p_k^{\alpha_k},\:n\in\mathbb{Z}^+\implies \phi(n) = \prod_{i=1}^k(p_i^{\alpha_i}-p_i^{\alpha_i-1})		
				\end{array}\]
				Donde $p_i^{\alpha_i}$ hace referencia a la descomposición en primos de $n\in\mathbb{Z}^+$
\end{theorem}
\begin{theorem}[Euler-Fermat]
	\[\text{mcd}(a,n)=1,\:a,n\in\mathbb{Z},\:n>1\longrightarrow a^{\phi(n)}\equiv 1\Mod{n}\longrightarrow [a]_n^{-1}=[a^{\phi(n)-1}]_n\]
\end{theorem}

\section{Sistemas de congruencias}

\begin{theorem}[Teorema chino del resto]
	\[
		\begin{array}{l}
	\forall\:n_1,n_2,\dots,n_k\in\mathbb{Z}^+\setminus\{1\}\\\forall\:a_1,a_2,\dots,a_k\in\mathbb{Z}\end{array}:\]
	\[\exists\:x/\left\{\begin{array}{c}
		x\equiv a_1\Mod{n_1}\\
		x\equiv a_2\Mod{n_2}\\
		\vdots\\
		x\equiv a_k\Mod{n_k}
	\end{array}\right\}
	\Longleftarrow \text{mcd}(n_i, n_j)=1,\:\forall\:i,j\in\{1,\dots,k\}\subseteq\mathbb{N},\:i\neq j
\]
Además:
\[x,x'\text{ son soluciones }\implies x\equiv x'\Mod{\prod_{i=1}^kn_i}\]
\end{theorem}

\begin{theorem}
	\[\begin{array}{l}
		\forall\:n_1, n_2\in\mathbb{Z}^+\setminus\{1\}\\
		\forall\:a_1,a_2\in\mathbb{Z}
		\end{array}:\:\exists x/\left\{\begin{array}{c}
			x\equiv a_1\Mod{n_1}\\
	x\equiv a_2\Mod{n_2}\end{array}
\right\}\iff a_1\equiv a_2\Mod{\text{mcd}(n_1,n_2)}
	\]
\end{theorem}


\part{Combinatoria}
\section{Conteo de conjuntos}
\begin{obs} $|A|<\infty>|B|:$
	\begin{enumerate}
		\item $|A\cup B| = |A| + |B| + |A\cap B|$.
		\item $B\subseteq A \implies |B|\leq|A|,\:|A\setminus B| = |A|-|B|$.
		\item $|A\times B| = |A||B|$.
		\item Principio de Palomar: $|A|>|B|\implies\nexists\:f:B\to A\:/\:f $ es inyectiva.
		\end{enumerate}
\end{obs}
\begin{obs}
	Las observaciones $4.1.2$ y $4.2.4$ se pueden generalizar a:
	\begin{enumerate}
		\item $|A\cup B \cup C| = |A|+|B|+|C|-|A\cap B| - |A\cap C| - |B\cap C| + |A\cap B\cap C|$ (imposible de generalizar).
		\item $|\prod_{i=1}^{\scriptscriptstyle |A|}A_i| = \prod_{i=1}^{\scriptscriptstyle |A|}|A_i|$
	\end{enumerate}
\end{obs}

\section{Variaciones}
\begin{definition}
	$V_{m,n}$: Variación odrinaria sin repetición de $m$ elementos tomados de $n$ en $n$ ($m\geq n$).
	\[V_{m,n} = \frac{m!}{(m-n)!} = \prod_{i=0}^{n+1}(m-i)\]
	\begin{itemize}
		\item No entran todos los elementos.
		\item Importa el orden.
		\item No se repiten los elementos.
	\end{itemize}
\end{definition}
\begin{definition}
	$VR_{m,n}$: Variación ordinaria con repetición de m elementos tomados de n en n.
	\[VR_{m,n} = m^n\]
\begin{itemize}
	\item Pueden entrar todos los elementos si $m\leq n$.
	\item Importa el orden.
	\item Se repiten los elementos.
	\end{itemize}
\end{definition}

\section{Permutaciones}
\begin{definition}
Las permutaciones son un caso particular de variaciones donde $m=n$.
\end{definition}
\begin{definition}
$P_n$: Permutación de $n$ elementos.
\[P_n = n!\]
\begin{itemize}
	\item Entran todos los elementos.
	\item Importa el orden.
	\item No se repiten los elementos.
	\end{itemize}
\end{definition}
\begin{definition}
	$PC_n$: Permutación circular. Los elementos se repetirán de forma cílcica, por lo cual, por ejemplo, la ordenación $1234$ sería equivalente a $3412$.
	\[PC_n = (n-1)!\]
\end{definition}
\begin{definition}
	$PR_n^{n_1,n_2,\dots,n_s}$: Permutaciones con repeticiones de $n$ elementos, donde hay $s$ elementos que se repiten con $n_i>1\:\forall\:i$.
	\[PR_n^{n_1,n_2,\dots,n_s}=\frac{n!}{\prod_{i=1}^s n_i!}\]
\end{definition}
\section{Combinaciones}
\begin{definition}
Las combinaciones son variaciones donde el orden no importa.
\end{definition}
\begin{definition}
	$C_{m,n}$: Combinación de $m$ elementos tomados de $n$ en $n$ ($m\geq n$).
	\[C_{m,n} = \arraycolsep=1.4pt\left(\begin{array}{c}
	m\\n
\end{array}\right) = \frac{m!}{n!(m-n)!}
\]
\begin{itemize}
	\item No entran todos los elementos.
	\item No importa el orden.
	\item No se repiten los elementos.
\end{itemize}
\end{definition}
\begin{definition}
	$CR_{m,n}$: Combinación con repetición de $m$ elementos tomados de $n$ en $n$ ($m\geq n$).
	\[CR_{m,n} = \arraycolsep=1.4pt\left(\begin{array}{c}m+n-1\\n\end{array}\right)\]

\end{definition}

\part{Teoría de grafos}
\section{Grafos, digrafos y multigrafos}
\begin{definition}
	Un grafo simple se denota como un par $G=(E,V)$ donde $E$ se refiere a las aristas y $V$ a los vértices. Las aristas se definen como:
	\[E\subseteq\{\{u,v\}: u,v\in V,\: u\neq v\}\]
\end{definition}
\begin{definition}
	Un multigrafo o grafo no dirigido $G=(V,E)$ tiene $E = \{e_i\}_{i\in I}$, que denota una familia, no un conjunto. Además, $e_i = \{u_i,v_i\},\:(u_i, v_i)\in V\times V\:\forall i\in I$. 
\end{definition}
\begin{definition}
	Un digrafo $G=(V,E)$ tiene $E\subseteq \{(a,b)\in V\times V : a\neq b\}$, o sea, las aristas son ordenadas y los pares indican la dirección de estas.
\end{definition}
\begin{definition}
Un multigrafo o grafo dirigido $G=(V,E)$ tiene $E=\{e_i\}_{i\in I}$ donde $e_i\in V\times V$.
\end{definition}
\begin{definition}
$u$ adyacente a $v \iff \{u,v\}\in E$, donde $G=(V,E)$. Entonces, $e=\{u,v\}$ conecta $u$ y $v$ quiere decir que $e$ incidente con $u$ y $v$ y que $u$ y $v$ son los extremos de la arista $e$.
\end{definition}
\begin{definition}
	$gr(u)$ donde $u\in V$ se refiere al número de aristas a las que pertenece $u$. $gr(u) = 0 \longrightarrow u$ es un vértice aislado.
\end{definition}
\begin{theorem} Para $G=(V,E)$ no dirigido: 
	\[\sum_{v\in V}gr(v) = 2|E|\]
\end{theorem}
\begin{definition}
	En un $G=(V,E)$ dirigido, para $(u,v)\in E$: $u$ es el vértice inicial de $(u,v)$ y $v$ el final. Sabiendo esto, se define el grado de entrada $gr^+(u)$ como el número de aristas con vértice final $u$. $gr^-(u)$ será el grado de salida y se referirá al número de aristas con $u$ como vértice inicial.
\end{definition}
\begin{theorem}
	Para $G=(V,E)$ dirigido:
	\[|E| = \sum_{v\in V}gr^+(v) = \sum_{v\in V}gr^-(v)\]
\end{theorem}
\section{Isomorfismo de grafos}
\begin{definition}
	Para $G_1=(V_1,E_1)$ y $G_2=(V_2,E_2)$ simples, $(G_1,G_2)$ isomorfos $\iff \exists\:f:V_1 \to V_2$ biyectiva $/\:\forall\:u,v\in V_1 : \{u,v\}\in E_1 \iff \{f(u),f(v)\}\in E_2$. En ese caso, se dice que $f$ es un isomorfismo de $(G_1,G_2)$. Esta definición se puede extender a multigrafos y multidigrafos.
\end{definition}
\begin{obs}\phantom{}
\begin{enumerate}
	\item Para $(G_1,G_2)$ isomorfos, se cumple $|V_1| = |V_2|$ y $|E_1| = |E_2|$.
	\item Para $f$ isomorfismo de $(G_1, G_2)$, se cumple $gr(u) = gr(f(u))\:\forall\:u\in V_1$.
\end{enumerate}
\end{obs}
\section{Árboles}
\begin{definition}
	Un camino entre $v_0$ y $v_k$ de $G=(V,E)$ no dirigido es una secuencia de vértices $C=(v_0,v_1,\dots,v_k)$ no necesariamente distintos. Se cumple que $e_i=\{v_{i-1},v_i\}\in E\:\forall\:i\in\{1,2,\dots,k\}$. $C$ es un ciclo $\iff v_0 = v_k \land v_i\neq v_j\:\forall\:i,j\in\{1,2,\dots,k-1\}$.
\end{definition}
\begin{definition}
	$G=(V,E)$ conexo $\iff\exists\:C\:\forall\:(v,u)\in V\times V,\:v\neq u$.
\end{definition}
\begin{definition}
	$G=(V,E)$ es un arbol $\iff G$ es conexo, sin ciclos y no dirigido. $G$ no tiene cilclos $\implies G$ es un grafo simple.
\end{definition}
\begin{theorem}
$G=(V,E)$ es un arbol $\iff$
\begin{enumerate}
\item $G$ es conexo $\land\:|V| = |E| +1$.
\item $G$ no tiene ciclos $\land\:|V| = |E| +1$.
\item $\exists!\:C\:\forall\:(v,u)\in V\times V$.
\item $G$ es conexto y al suprimir cualquier arista $G$ pasa a ser no conexto$\implies G$ es un grafo conexo minimal.
\end{enumerate}
\end{theorem}

\begin{definition}
	$G'=(V',E')$ es subgrafo de $G=(V,E) \iff V'\subseteq V\:\land\:E'\subseteq E$.
\end{definition}
\begin{definition}
	$G'=(V',E')$ es un árbol generador de $G=(V,E)\iff G'$ subgrafo $G\:\land\:V'=V$.
\end{definition}
\begin{theorem}
	$G=(V,E)$ simple es conexo $\iff \exists\:$ abrol generador de $G$.
\end{theorem}
















\end{document}
